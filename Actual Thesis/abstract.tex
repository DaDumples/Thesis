In the past, analyses of lightcurve data have been applied to asteroids in order to determine their axis of rotation, rotation rate and other parameters. In recent decades, these analyses have begun to be applied in the domain of Earth orbiting spacecraft. Due to the complex geometry of spacecraft and the wide variety of parameters that can influence the way in which they reflect light, these analyses require more complex assumptions and a greater knowledge about the object being studied. Previous investigations have shown success in extracting attitude parameters from unresolved spacecraft using simulated data. This paper presents a focused attempt to derive attitude parameters using an Unscented Kalman Filter from both simulated and real data provided by Lockheed Martin Space.

This thesis characterizes and presents the differences in performance between three simulated geometries in low, medium, and geostationary orbit in both cases where they are spinning about a constant axis and in cases in which they are tumbling. 

Additionally, this thesis hypothesizes and tests the idea that a predictable and extraneous angular velocity solution exists which is the reflection of the true solution about the plane defined by the sun and observation vectors. This thesis encountered multiple instances of this type solution appearing in simulation and provides an example as well as a visualization.

Finally, this thesis demonstrates the ability to converge to a solution from real data although there were large discrepancies between the measurement model and the data. This thesis discusses the validity of these solutions and sources of error.