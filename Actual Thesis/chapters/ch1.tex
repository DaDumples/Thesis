\chapter*{Introduction}
This is supposed to be chapter1?In the past, analyses of lightcurve data have been applied to asteroids in order to determine their axis of rotation, rotation rate and other parameters. In recent decades, these analyses have begun to be applied in the domain of Earth orbiting spacecraft. Due to the complex geometry of spacecraft and the wide variety of parameters that can influence the way in which they reflect light, these analyses require more complex assumptions and a greater knowledge about the object being studied. Previous investigations have shown success in extracting attitude parameters from unresolved spacecraft using simulated data. This paper presents a focused attempt to derive attitude parameters using an Unscented Kalman Filter from real, high quality lightcurve data provided by Lockheed Martin's Santa Cruz observation site. This paper describes methods of modeling spacecraft reflections using a PHONG brdf in combination with an approximate ray tracing algorithm. In addition, this paper describes the results across various simulated scenarios and comparisons between the simulations and applications to real data. Finally, this paper describes the challenges associated with creating a physically accurate model of spacecraft reflectance and suggests how future work might improve this process.In the past, analyses of lightcurve data have been applied to asteroids in order to determine their axis of rotation, rotation rate and other parameters. In recent decades, these analyses have begun to be applied in the domain of Earth orbiting spacecraft. Due to the complex geometry of spacecraft and the wide variety of parameters that can influence the way in which they reflect light, these analyses require more complex assumptions and a greater knowledge about the object being studied. Previous investigations have shown success in extracting attitude parameters from unresolved spacecraft using simulated data. This paper presents a focused attempt to derive attitude parameters using an Unscented Kalman Filter from real, high quality lightcurve data provided by Lockheed Martin's Santa Cruz observation site. This paper describes methods of modeling spacecraft reflections using a PHONG brdf in combination with an approximate ray tracing algorithm. In addition, this paper describes the results across various simulated scenarios and comparisons between the simulations and applications to real data. Finally, this paper describes the challenges associated with creating a physically accurate model of spacecraft reflectance and suggests how future work might improve this process.In the past, analyses of lightcurve data have been applied to asteroids in order to determine their axis of rotation, rotation rate and other parameters. In recent decades, these analyses have begun to be applied in the domain of Earth orbiting spacecraft. Due to the complex geometry of spacecraft and the wide variety of parameters that can influence the way in which they reflect light, these analyses require more complex assumptions and a greater knowledge about the object being studied. Previous investigations have shown success in extracting attitude parameters from unresolved spacecraft using simulated data. This paper presents a focused attempt to derive attitude parameters using an Unscented Kalman Filter from real, high quality lightcurve data provided by Lockheed Martin's Santa Cruz observation site. This paper describes methods of modeling spacecraft reflections using a PHONG brdf in combination with an approximate ray tracing algorithm. In addition, this paper describes the results across various simulated scenarios and comparisons between the simulations and applications to real data. Finally, this paper describes the challenges associated with creating a physically accurate model of spacecraft reflectance and suggests how future work might improve this process.In the past, analyses of lightcurve data have been applied to asteroids in order to determine their axis of rotation, rotation rate and other parameters. In recent decades, these analyses have begun to be applied in the domain of Earth orbiting spacecraft. Due to the complex geometry of spacecraft and the wide variety of parameters that can influence the way in which they reflect light, these analyses require more complex assumptions and a greater knowledge about the object being studied. Previous investigations have shown success in extracting attitude parameters from unresolved spacecraft using simulated data. This paper presents a focused attempt to derive attitude parameters using an Unscented Kalman Filter from real, high quality lightcurve data provided by Lockheed Martin's Santa Cruz observation site. This paper describes methods of modeling spacecraft reflections using a PHONG brdf in combination with an approximate ray tracing algorithm. In addition, this paper describes the results across various simulated scenarios and comparisons between the simulations and applications to real data. Finally, this paper describes the challenges associated with creating a physically accurate model of spacecraft reflectance and suggests how future work might improve this process.In the past, analyses of lightcurve data have been applied to asteroids in order to determine their axis of rotation, rotation rate and other parameters. In recent decades, these analyses have begun to be applied in the domain of Earth orbiting spacecraft. Due to the complex geometry of spacecraft and the wide variety of parameters that can influence the way in which they reflect light, these analyses require more complex assumptions and a greater knowledge about the object being studied. Previous investigations have shown success in extracting attitude parameters from unresolved spacecraft using simulated data. This paper presents a focused attempt to derive attitude parameters using an Unscented Kalman Filter from real, high quality lightcurve data provided by Lockheed Martin's Santa Cruz observation site. This paper describes methods of modeling spacecraft reflections using a PHONG brdf in combination with an approximate ray tracing algorithm. In addition, this paper describes the results across various simulated scenarios and comparisons between the simulations and applications to real data. Finally, this paper describes the challenges associated with creating a physically accurate model of spacecraft reflectance and suggests how future work might improve this process.In the past, analyses of lightcurve data have been applied to asteroids in order to determine their axis of rotation, rotation rate and other parameters. In recent decades, these analyses have begun to be applied in the domain of Earth orbiting spacecraft. Due to the complex geometry of spacecraft and the wide variety of parameters that can influence the way in which they reflect light, these analyses require more complex assumptions and a greater knowledge about the object being studied. Previous investigations have shown success in extracting attitude parameters from unresolved spacecraft using simulated data. This paper presents a focused attempt to derive attitude parameters using an Unscented Kalman Filter from real, high quality lightcurve data provided by Lockheed Martin's Santa Cruz observation site. This paper describes methods of modeling spacecraft reflections using a PHONG brdf in combination with an approximate ray tracing algorithm. In addition, this paper describes the results across various simulated scenarios and comparisons between the simulations and applications to real data. Finally, this paper describes the challenges associated with creating a physically accurate model of spacecraft reflectance and suggests how future work might improve this process.In the past, analyses of lightcurve data have been applied to asteroids in order to determine their axis of rotation, rotation rate and other parameters. In recent decades, these analyses have begun to be applied in the domain of Earth orbiting spacecraft. Due to the complex geometry of spacecraft and the wide variety of parameters that can influence the way in which they reflect light, these analyses require more complex assumptions and a greater knowledge about the object being studied. Previous investigations have shown success in extracting attitude parameters from unresolved spacecraft using simulated data. This paper presents a focused attempt to derive attitude parameters using an Unscented Kalman Filter from real, high quality lightcurve data provided by Lockheed Martin's Santa Cruz observation site. This paper describes methods of modeling spacecraft reflections using a PHONG brdf in combination with an approximate ray tracing algorithm. In addition, this paper describes the results across various simulated scenarios and comparisons between the simulations and applications to real data. Finally, this paper describes the challenges associated with creating a physically accurate model of spacecraft reflectance and suggests how future work might improve this process.In the past, analyses of lightcurve data have been applied to asteroids in order to determine their axis of rotation, rotation rate and other parameters. In recent decades, these analyses have begun to be applied in the domain of Earth orbiting spacecraft. Due to the complex geometry of spacecraft and the wide variety of parameters that can influence the way in which they reflect light, these analyses require more complex assumptions and a greater knowledge about the object being studied. Previous investigations have shown success in extracting attitude parameters from unresolved spacecraft using simulated data. This paper presents a focused attempt to derive attitude parameters using an Unscented Kalman Filter from real, high quality lightcurve data provided by Lockheed Martin's Santa Cruz observation site. This paper describes methods of modeling spacecraft reflections using a PHONG brdf in combination with an approximate ray tracing algorithm. In addition, this paper describes the results across various simulated scenarios and comparisons between the simulations and applications to real data. Finally, this paper describes the challenges associated with creating a physically accurate model of spacecraft reflectance and suggests how future work might improve this process.In the past, analyses of lightcurve data have been applied to asteroids in order to determine their axis of rotation, rotation rate and other parameters. In recent decades, these analyses have begun to be applied in the domain of Earth orbiting spacecraft. Due to the complex geometry of spacecraft and the wide variety of parameters that can influence the way in which they reflect light, these analyses require more complex assumptions and a greater knowledge about the object being studied. Previous investigations have shown success in extracting attitude parameters from unresolved spacecraft using simulated data. This paper presents a focused attempt to derive attitude parameters using an Unscented Kalman Filter from real, high quality lightcurve data provided by Lockheed Martin's Santa Cruz observation site. This paper describes methods of modeling spacecraft reflections using a PHONG brdf in combination with an approximate ray tracing algorithm. In addition, this paper describes the results across various simulated scenarios and comparisons between the simulations and applications to real data. Finally, this paper describes the challenges associated with creating a physically accurate model of spacecraft reflectance and suggests how future work might improve this process.