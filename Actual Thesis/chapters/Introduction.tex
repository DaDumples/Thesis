\chapter{Introduction}

Light curve data is possible one of the most accessible and plentiful sources of data that can be acquired from a spacecraft. It requires nothing from the spacecraft and can be collected by anyone for as many passes as could be desired. Given that interest in utilizing Earth orbit has never been higher, the utility of this rich vein of data has also never been higher.

Because the light reflected from a spacecraft is so dependent on that spacecrafts geometry and attitude it is possible to make estimations about one or the other. This thesis focuses on the latter, but substantial work has already been published on the topic of geometry estimation.

Being able to estimate the attitude of a spacecraft for the duration of a pass also enables the estimation of its angular velocity, and these two quantities combined offer incredible potential to increase the situational awareness of space operations. For example, recently a well funded effort to deorbit space debris has begun both by government and commercial organizations. This undertaking must contend with the challenge that their targets are often spinning and telemetry is unavailable. Here, light curve data enables the spin of the target to be estimated without requiring telemetry and allows for operators to be prepared before proximity operations with the target begin.

Additionally, it offers a method of identifying and classifying uncatalogued objects. Because attitude and spin are so closely linked to the mission of a spacecraft, estimating these from light curve data allows for debris to be differentiated from functioning spacecraft and more.

In this thesis, attempts to bring the theoretical prospects of lightcurve data into practical ones. Using high quality data provided by Lockheed-Martins Santa Cruz observation site, attitude and spin rate estimation will be attempted using a formulation of Unscented Kalman Filter.