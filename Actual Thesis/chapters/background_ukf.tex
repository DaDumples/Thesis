\chapter*{Introduction}

\subsection{Unscented Kalman Filter}
The Kalman Filter is a statistical estimation technique derived with the intention of solving state estimation problems. \cite{ukf_merwe} The derivation for the Kalman filter assumes that the system is linear and for this case is shown to be optimal \cite{kf_kalman}. However, many estimation problems exist which are nonlinear in nature and for these cases modification to the original Kalman Filter have been made. These modifications are the Extended and Unscented formulations which approach the problem in different ways. In this thesis the Extended formulation will be mentioned but not explored in depth as it can quickly be shown to be unsuitable for the application of this thesis.

The Kalman Filter in a sentence would be the following: The Kalman Filter combines a model of how a system is expected to behave and compares the predictions of this model with real world measurements of the system to discover what state of the system best corroborates the two. In other words, the Kalman Filter attempts to minimize the difference between measured data and predicted data by finding the state estimate that minimizes the difference \cite{kf_derivation}.

Formulating a Kalman Filter begins by defining the model and measurements of a system to be the following:
\begin{align}
\dot{x} = f(x) + w_x \\
z = h(x) + w_z
\end{align}
Where $x$ is the state of the system, $z$ is the measurement, $w_x$ and $w_z$ are the noise for the system and measurement models respectively. $f(x)$ describes the dynamics of the system and $h(x)$ is the measurement function. $w_x$ and $w_z$ are assumed to be gaussian with a mean of zero \cite{kf_derivation}. 

If the system is completely linear, as it is assumed to be in the standard Kalman Filter, these functions can be written in the following form:
\begin{align}
\dot{x} = Ax + w_x \\
z = Hx + w_z
\end{align}
Where $A$ is now the system dynamics matrix and $H$ is the measurement matrix.

Since $w_x$ is assumed ot have a mean of zero, the differential equation for $x$ can be solved by the equation
\begin{align}
x(t) = x_0e^{At}
\end{align}
and thus:
\begin{align}
x(t + \delta t) = x(t)e^{A\delta t}
\end{align}
By plugging $A\delta t$ into the series expansion of the exponential function you can calculate the state transition matrix $F$ such that
\begin{align}\label{predict_step}
x_{k+1} = Fx_{k}
\end{align}
Where $k$ denotes an discrete measurement time evenly spaced by $\delta t$.

The matrices $F$ and $H$ are the crux of the Kalman Filter and can only truly be calculated when the system and measurement functions are linear. 

The final major component of the Kalman Filter is the state error covariance matrix, $P$ which is defined to be \cite{kf_derivation}
\begin{align}
P_k = E[(\hat{x}_k - x_k)(\hat{x}_k - x_k)^T]
\end{align} 
with $\hat{x}_k$ defined to be the true state, which can never truly be known. Because of this uncertainty, the covariance matrix must also be estimated and propagated with the state. The propagation equation for $P$ is the following:
\begin{align}\label{predict_p}
P_{k+1} = FP_{k-1}F + Q
\end{align} 
where $Q$ describes the error covariance of the system model and is defined to be
\begin{align}
Q = E[w_xw_x^T].
\end{align} 

At each timestep the state estimate, $x_k$, is updated using the following equation:
\begin{align}
x_{k} = x'_{k} + K_k(z_k - Hx'_k)
\end{align} 
where $x'_k$ is the predicted state from equation \eqref{predict_step} and $K_k$ is called the Kalman gain and is calculated using
\begin{align}
K_k = P'_kH^T(HP'_kH^T + R)^{-1}.
\end{align} 
$P'_k$ is the predicted covariance matrix calculated from equation \eqref{predict_p} and $R$ describes the error covariance of the sensors and is defined to be
\begin{align}
R = E[w_zw_z^T].
\end{align} 
Finally, the state error covariance matrix, $P$, is also updated using
\begin{align}
P_k = (I - K_kH)P'_k
\end{align} 
The final process can be summarized in the following table.

\begin{center}
\begin{tabular}{ | m{5em} | m{1cm}| } 
	\hline
	Predict & \begin{align*} x'_{k} = Fx_{k-1} \\ P'_{k} = FP_{k-1}F + Q \end{align*}\\ 
	\hline
	Update & \begin{align*} K_{k} = P'_{k}H^T(HP'_{k}H^T + R)^{-1} \\ x_{k} = x'_{k} + K_k(z_k - Hx'_k) \\ P_k = (I - K_kH)P'_k \end{align*}\\ 
	\hline
\end{tabular}
\end{center}
