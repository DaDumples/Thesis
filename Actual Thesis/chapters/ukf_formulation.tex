\chapter{Unscented Kalman Filter Formulation}

\section{assumptions}

This thesis assumes knowledge of the spacecraft geometry. This thesis attempted to derive a UKF which required data that could be plausibly acquired in an analysis of real data. Unfortunately, this problem requires either the geometry or the angular velocity to be known as one cannot be determined without the other. This thesis assumes that the geometry is accessible, either from photographs, access to design drawings or other means.

This thesis does not assume any prior knowledge of attitude, angular velocity, or inertia of the object. The first formulation described is for the case in which the inertia does not need to be known which occurs when the object is spinning about its major axis. The second includes the inertia in the state estimate to enable attitude estimation in the case of a tumbling object.

\section{Non-Tumbling Formulation}
This thesis actually utilizes two formulations of the UKF. The defines the state vector is composed of the modified Rodriguez parameters and angular velocity and is in the form:
\begin{equation}
\bm{x} = \begin{bmatrix} \bm{p} \\ \bm{\omega} \end{bmatrix}
\end{equation}
where $\bm{p}$ are the Modified Roriguez Parameters (MRPs) and  $\boldmath{w}$ is the angular velocity vector. This formulation is utilized in this thesis to filter data for objects that are assumed to be spinning about a constant axis or whose inertial properties are known. In either of these cases inertial properties do not need to be estimated.

The Modified Rodriguez Parameters were selected as the attitude representation of choice for this thesis. The use of quaternions and Euler Angles were investigated but were not used for various reasons. Quaternions suffer from the issue that because they over define an attitude, singularities arise in the UKF covariance matrix unless complex additional measures are taken \cite{Linares_data_fusion}. Additionally, values cannot be added to a quaternion as it is unit norm constrained which poses an extra challenge during the update step of the UKF.

Euler angles do not suffer from the singular covariance issue as they are a minimal attitude representation and are not unit norm constrained, however they do suffer from another form of singularity where in certain orientation two of the angles become indeterminate and can vary wildly with only small changes in attitude. This makes it very difficult for the UKF to converge on a solution.

Modified Rodriguez Parameters are related to the quaternion, but they represent the attitude in a minimal form using only three elements. This means that they do not result in a singular covariance matrix. Additionally, they are not unit norm constrained and only have a single singularity near the attitude represented by the quaternion as $-1 + 0i + 0j + 0k$ where the MRP elements tend towards infinity. This requires a fix, but one which requires no conditions. Fortunately, for each MRP there is an equivalent, known as a shadow parameter, which has represents the same attitude but whose magnitude is inverted \cite{MRPs}. This allows you to swap between the MRP and its "shadow parameter" whenever its magnitude gets too large. This maintains a consistent attitude parametrization without ever getting near the singularuty. The equation for converting between an MRP and its shadow parameter and vice verse is the following:

\begin{equation}
\bm{p}_{shadow} = -\frac{\bm{p}}{||\bm{p}||^2}
\end{equation}

The kinetmatic equations of motion for the Modified Rodriguez Parameters are as follows \cite{Crassidis_MRPs}:

\begin{equation}
	\dot{\bm{p}} = \frac{1}{2} \left ( \frac{1}{2} (1 - \bm{p}^T\bm{p})I + \bm{p}^{\times} + \bm{p}\bm{p}^T \right )\bm{\omega}
\end{equation}

\section{Tumbling Object Formulation}

In the case that the object is tumbling, that is to say that the angular velocity is not constant in the ECI frame, inertial properties need to be either known or estimated. If these properties are known then the state definition above suffices and only the forward propagation model must be changed. This is unlikely however as it is often difficult even for operators to accurately measure a spacecrafts inertial properties. This leads to the necessity to estimate the inertial properties.

The simplest approach to estimating the inertial properties is to take advantage of the symmetric properties of the inertia matrix and add the matrices diagonal and either its upper or lower triangular. This adds six elements to the state that add computational complexity and increase the difficulty for the UKF to converge on accurate values. This thesis makes the following two assumptions to reduce the number of elements that are required to define the inertial properties.
\begin{enumerate}
\item The principal inertial frame is sufficiently aligned to the frame in which the geometry is defined such that the off-diagonal elements are negligible. This allows for the off-diagonal elements to be assumed to be zero, reducing the elements required to be estimated to only the diagonal.

\item The disturbance torques applied to the spacecraft are negligible over the duration of a single pass. In can be shown that in the absence of disturbing torques the specific values of the inertia matrix no longer matter, rather it is only their relative magnitudes. This assumption is valid a wide range of spacecraft as the duration of a single pass is so small that very few torques could significantly affect measurements. This assumption allows for one element of the inertia matrix to be assumed to be one.
\end{enumerate}
The combination of these two assumptions reduces the number of elements that needed to be estimated to two: the second and third diagonal elements. Given these assumptions, the state vector becomes:
\begin{equation}
\bm{x} = \begin{bmatrix} \bm{p} \\ \bm{\omega} \\ \bm{\Xi} \end{bmatrix}
\end{equation}
Where $\bm{\Xi}$ is defined to be a 2x1 vector containing two of the diagonal elements of the objects principal inertia matrix.

\section{Modifications}

The only midification from the standard UKF implementation used in this thesis was to bound the angular velocity and inertia estimates between reasonable values. In preliminary simulations it was found that the UKF would initially deviate significantly before settling on a value. This large deviation both slowed down computation significantly and resulted in the UKF settling on unrealistically high angular velocities.